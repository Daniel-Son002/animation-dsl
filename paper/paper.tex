\documentclass[acmsmall,screen]{acmart}
\setcopyright{none}
\copyrightyear{2022}
\acmYear{2022}
\acmDOI{}

\citestyle{acmauthoryear}

\usepackage{fancyhdr}
\usepackage{float}
\usepackage{listings}
\usepackage{caption}
\DeclareCaptionFont{white}{\color{white}}
\DeclareCaptionFormat{listing}{%
  \parbox{\textwidth}{\colorbox{gray}{\parbox{\textwidth}{#1#2#3}}\vskip-4pt}}
\captionsetup[lstlisting]{format=listing,labelfont=white,textfont=white}
\lstset{frame=lrb,xleftmargin=\fboxsep,xrightmargin=-\fboxsep}

\begin{document}

\title{Animato: A Domain-Specific Language for Animation and Drawing}

\author{Daniel Son}
\email{danielson@college.harvard.edu}
\affiliation{%
  \institution{Harvard University}
  \country{USA}
}

\renewcommand{\shortauthors}{Daniel Son}

\settopmatter{printacmref=false}
\settopmatter{printfolios=true}
\renewcommand\footnotetextcopyrightpermission[1]{}
\pagestyle{fancy}
\fancyfoot{}
\fancyfoot[R]{CS252R (Fall 2024)}
\fancypagestyle{firstfancy}{
  \fancyhead{}
  \fancyhead[R]{CS252R (Fall 2024)}
  \fancyfoot{}
}
\makeatletter
\let\@authorsaddresses\@empty
\makeatother

\begin{abstract}
    Animato is a domain-specific language aimed to tackle the problem of tedious drawing for animation and shape rendering.
    While there exists tools such as p5.js~\cite{p5js} and Penrose~\cite{penrose}, this project mimics and extends certain aspects of existing tools.
    We design a domain-specific language that takes shape inputs with locations and outputs the drawings on a canvas as a \texttt{.png} file.
    Depending on the input, the output may be a singular or multiple image files where the sequence of image files can be input into an animation software such as Adobe Premiere Pro or TV Paint to create an animation.
\end{abstract}

% \keywords{music synthesis, first species counterpoint, domain specific language, programming languages}

\maketitle
\thispagestyle{firstfancy}

\section{Introduction}
Art is prevalent in all shapes and forms, one being visual.
In the field of animation, drawing is a tedious yet important task that 2-dimensional animators often have to utilize.
Due to the tedious nature of hand-drawn animation, Animato aims to relieve animators of the need to draw each frame and develop animations using the domain-specific language.

\subsection{Related Work}
Previously, much work has been done in the p5.js \cite{p5js} community as well as the Penrose \cite{penrose} CMU community to advance the code in art space. 
p5.js is an open-source JavaScript library that aims to make creative coding accessible to artists and coders alike. 

\section{Framework}
\subsection{Primitives}
In the current iteration of Animato, there are multiple primitives for shapes that are ready to be utilized.
These include squares, circles, line segments, and some other harder to achieve common shapes. 

\subsection{Implementation}
Animato uses a Python backend with Pillow for drawing visualizations and z3 for constraint solving.

\section{Example}
Using these primitives, users are able to create custom shapes (functions) by defining a function as follows:

\begin{lstlisting}[label=code, caption=Sample custom function definition in Animato]
    dsl_code = """
      (define star (cx 0) (cy 0) (radius 50) (
        (line (x1 (- cx radius))
              (y1 (- cy radius))
              (x2 (+ cx radius))
              (y2 (+ cy radius)))
        (line (x1 (- cx radius))
              (y1 (+ cy radius))
              (x2 (+ cx radius))
              (y2 (- cy radius)))
        (line (x1 cx) (y1 (- cy radius))
              (x2 cx) (y2 (+ cy radius)))
        (line (x1 (- cx radius)) (y1 cy)
              (x2 (+ cx radius)) (y2 cy))
      ))
      (star (cx 250) (cy 250) (radius 100))
    """
\end{lstlisting}

\begin{center}
  \begin{figure}[!h]
    \includegraphics[width=0.5\textwidth]{images/star.png}
    \caption{Output image of \texttt{dsl\_code} input}
  \end{figure}
\end{center}

The above \texttt{dsl\_code} input produces the drawing shown in figure 1.

\section{Limitations and Future Work}
While the intent of this project is to create a domain-specific language to assist programmers and animators to easily draw animations, there is still work that needs to be done in order to flesh out this idea.
Firstly, the aspect of custom-defined functions is quite limited and Animato needs more functionality that gives a more user-friendly experience.
While the aim of Animato is to allow everyone to create animations without the need to look in the backend to see what it's doing, this is quite difficult to allow users to actually do whatever they want.
In testing, it has been apparent that actually creating custom functions is quite hard since the user needs a precise understanding of the canvas space.

\section{Conclusion}
Overall, while the aim of \textit{Animato} is to provide a more concise and specialized language for animators/developers to use, further work must be done to polish the domain-specific langauge.
At the current state, users are able to create a canvas with some specified shape drawings, including custom shapes. 

\begin{acks}
This paper and project serves as my final project for CS 252, Advanced Programming Languages, at Harvard University. Special thanks to Professor Nada Amin, Raffi Sanna, and classmates in CS 252 for providing feedback and advice on the project in it's developmental stage.
\end{acks}

\bibliography{references}
\bibliographystyle{ACM-Reference-Format}

\end{document}